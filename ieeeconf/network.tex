\documentclass[letterpaper, 10 pt, conference]{ieeeconf}
\IEEEoverridecommandlockouts
\overrideIEEEmargins

\usepackage[pdftex]{graphicx}
\usepackage{amsmath}

\title{\LARGE \bf
 Theoretical and Numerical Studies on Percolation Mechanism of Carbon Nanotube Network}

\author{Lei Wen$^1$, Min Zhang$^2$
  \thanks{*Research supported by Shenzhen competitive research grants KQCX20130628093909155 and National Nature Science Foundation of China 61504004. This work was conducted in Shenzhen TFT and Advanced Display Lab.}
  \thanks{$^{1}$Lei Wen is with School of Electronic and Computer Engineering, Peking University, Shenzhen, 518055, China (email: wenlei@sz.pku.edu.cn).}
  \thanks{$^{2}$Min Zhang is with School of Electronic and Computer Engineering, Peking University, Shenzhen, 518055, China (corresponding author, phone: 0755-26032482; fax: 0755-26032015; email: zhangm@ece.pku.edu.cn)}
}

\begin{document}

\maketitle
\thispagestyle{empty}
\pagestyle{empty}

\begin{abstract}
  In this paper, a theoretical and Monte-Carlo analysis on the percolation behavior of single-walled carbon nanotube (SWCNT) network are proposed. For such network structures that are commonly used as the conduction channel in thin-film transistors (TFTs), the change of channel resistance $R_{ch}$ with varied channel length $R_{ch}$, tube density $D$, and average tube length $l_{cnt}$ is carefully studied. A linear relation between $R_{ch}$ and $L_{ch}$ is proved. A negative exponential dependence that $R_{ch}$ has on $D$ and $l_{cnt}$ is found through analytical deductions. The simulation work is implemented using MATLAB and HSPICE, in which a specific trimming algorithm is developed to iteratively cut off all useless branches of the network. All the simulation results show very good agreements with the theoretical expressions.
\end{abstract}

\section{introduction}

Single-walled carbon nanotube (SWNT) is a promising material in high performance thin-film transistors (TFT) due to its unique properties such as high mobility [1]. It has also exhibited great compatibility in future flexible electronic applications [2]-[4]. For carbon nanotube thin-film transistors (CNT-TFTs), better current drivability is obtained with the increase of the network density. However, the metallic nanotubes can cause the circuit shorting in a density higher than the percolation threshold [5]. Hence, the physical and electrical properties of the network structure becomes an important research topic. There are two main aspects of the network behavior, the tube-tube crossing contacts and the percolation paths, since the whole network can be considered as a large scale circuit consisting of pure resistors, regardless of the CNT/metal contacts on the source/drain electrodes. The tube-tube contact conduction is believed to be a sort of tunneling effect. In a thin-insulating-film model [6], the general tunneling resistance is expressed as
\begin{equation}
  R_c=\frac{h^2d}{e^2A\sqrt{2m_e\varphi}}\exp(\frac{4\pi d\sqrt{2m_e\varphi}}{h})
\end{equation}
where $h$ is Plank's constant, $e$ is charge of electron, $d$ is distance between the two electrodes, $\varphi$ is height of the rectangular barrier, $m_e$ is static mass of electron, and $A$ is the contact area. This equation is widely accepted by many studies [7]-[10]. However, there is another model that describes the tube-tube tunneling resistance as
\begin{equation}
  R_c=\frac{h}{2Me^2}\exp(\frac{4\pi d\sqrt{2m_e\varphi}}{h})
\end{equation}
where M is the overall channel number of CNT, $M=400\sim 500$ for multi-walled CNT, and $M=2$ for SWNT [16].
Given that $\varphi=0.4$ eV as half bandgap of a SWNT with diameter of $d_{cnt}=1.1$ nm, $A=d_{cnt}^2$, and $d=0.34$ nm as the average Van der Waals separation [10], the calculated resistance of a single contact is $R_c=1.54$ M$\Omega$ for Eq. (1) and $R_c=1.17$ M$\Omega$ for Eq. (2).

Another aspect is the percolation mechanism of the network. There are three key geometrical parameters that are closely related to the channel percolation, the channel length $R_{ch}$, the tube density $D$ and the average tube length $l_{cnt}$. Some previous studies have found that the channel resistance of a CNT-TFT has an exponential relation on the channel length, $R_{ch}\propto(L_{ch})^m$, where the factor $m$, according to their results, is about 1.9 at most [10,13], and will gradually degrade to 1 when the density is getting higher than the percolation threshold ($Dl_{cnt}^2\geq4.236^2/\pi$ [16]). However, though several authors have tried to build a numerical model as a function of geometrical and physical parameters [10]-[15], their results are mostly confined to the effect of $L_{ch}$. The overall relations between the electrical properties and other parameters (i.e., $l_{cnt}$, $D$) is still unclear. In this paper, an analytical model is proposed by implementing both theoretical and numerical methods, providing a clear and direct understanding on the percolation behavior of the network with different geometrical parameters.

\section{theoretical analysis}

For a complex network with pure and linear resistors, the Kirchhoff's law is expressed as
\begin{equation}
  \left\{
  \begin{aligned}
    &\sum_m i_{cnt}+\sum_n i_{contact}=0 \\
    &\sum_m i_{cnt}r_{cnt}+\sum_ni_{contact}r_{contact}=0
  \end{aligned}
  \right.
\end{equation}
where m stands for the total number of CNT's intrinsic resistors and n for the tunneling contacts for any closed-loop path.

\begin{figure}[!h]
  \includegraphics[width=3.5in]{figure-1.jpg}
  \caption{(a) A schematic diagram about combining two same linear circuits into one. (b) A random network of CNT with the tube length of 1/$\mu$m and density of 10 $\mu$m$^{-2}$. The angle of every tube is evenly distributed from 0 to $2\pi$. The tube-tube contacts were found out from between each two of all the tubes, marked as red points.
  }
\end{figure}

According to homogeneous theorem, when the given bias voltage changes by $k$ times in such linear resistance circuit, the corresponding total current will change by $k$ times too. Considering a simple and regular network as shown in Fig. 1(a), by adding a same network in series with it, the total current that pass through node B and voltage drop between A and B will both reduce by half. And a new circuit of twice length is formed and the total resistance should be about $2R_0$ with little error caused by the change of boundary conditions in node B. For a much larger linear network in Fig. 1(b), no matter how complex it is, a certain cutting path that divide the total circuit into two pieces with the same total resistance can be found, and the expectation of the cutting path's position should be near the center. The boundary conditions' error can also be ignored. Hence, due to the linear resistance of tube-tube contacts, the whole network shows a linear relation on its channel length.
\begin{equation}
  R_{ch}\propto L_{ch}
\end{equation}

The analysis about behavior of $R_{ch}$ with varied $l_{cnt}$ is divided into two steps by introducing a new variable $D_c$ that represents the density of tube-tube crossing points. Assume that there is a tube $A$ with coordinates of $A_1(0,0)$ and $A_2(0,l_A)$, where $l_A$ is the tube length. Another tube $B$ with coordinates of $B_1(x_1,y_1)$, $B_2(x_2,y_2)$ and length of $l_B$ is randomly dropped on that surface with total area of $X\times Y$. The contact conditions is then expressed as
\begin{equation}
  \left\{
  \begin{aligned}
    &y_1y_2=y_1(y_1+l_B\sin\alpha)<0 \\
    &y_1\cot\alpha<x_1<y_1\cot\alpha+l_A
  \end{aligned}
  \right.
\end{equation}
where $-\frac{\pi}{2}<\alpha<\frac{\pi}{2}$ is the angle of tube $B$. The contact possibility can be calculated through a double integral
\begin{equation}
  \begin{aligned}
    p_c&=\frac{1}{\pi XY}\iint_Sl_A\mathrm{d}y\mathrm{d}\alpha \\
    &=\frac{2l_Al_B}{\pi XY}\int_0^{\frac{\pi}{2}}\sin(\alpha)\mathrm{d}\alpha=\frac{2l_Al_B}{\pi XY}
  \end{aligned}
\end{equation}
where S is a region defined by the first equation of Eq. (5) and is plotted in Fig. 2(b). For a TFT channel, specifically, the area of channel is $A=L_{ch}W$, the total number of tubes is $N=DA$. Thus the expectation number of contacts is
\begin{equation}
  \begin{aligned}
    N_c&=p_c+2p_c+3p_c+...+(N-1)p_c \\
    &=\frac{N(N-1)p_c}{2}\approx \frac{N^2p_c}{2}
  \end{aligned}
\end{equation}
Given the average tube length $l_{cnt}$, the density of contacts in the channel can be derived as
\begin{equation}
  D_c=\frac{N_c}{A}=\frac{N^2p_c}{2A}=\frac{2(DA)^2l_{cnt}^2}{2\pi A^2}=\frac{D^2l_{cnt}^2}{\pi}
\end{equation}

\begin{figure}[!h]
  \centering
  \includegraphics[width=3.4in]{figure-2.jpg}
  \caption{
    (a) Two tubes crossing over each other. (b) The integral region defined by $y_1(y_1+l_B\sin\alpha)<0$.
  }
\end{figure}

The next step is to find the relation between $R_{ch}$ and $D_c$. It is much tougher than the deduction of $D_c$, since the percolation paths in a random network are so irregular and nearly impossible to predict. However, a successful approach to a semi-quantitative results is achieved. As shown in Fig. 3, with the increase of tube length, more contacts will appear, providing extra parallel paths which can reduce the resistance significantly. It should be noted that almost every tube in the network has a non-horizontal angle and lengthening them means more connection to nearby resistors. In other words, the growth of $D_c$ will significantly cause the increase of parallel paths. Thus, the $R_{ch}$ falls rapidly and better transistor performance is obtained. The relation between $R_{ch}$ and $D_c$ could be written as
\begin{equation}
  R_{ch}\propto\frac{1}{D_c^{\lambda}}
\end{equation}
where the accurate value of $\lambda$ is very hard to get from pure theoretical analysis, however, from Fig. 3 it can be confirmed at least that $1<\lambda<2.7$. By combine Eq. (8) and Eq. (9) it can be derived that
\begin{equation}
  R_{ch}\propto\frac{1}{(Dl_{cnt})^{2\lambda}}
\end{equation}
A more accurate $\lambda$ is obtained through the Monte-Carlo simulations.

\begin{figure}[!h]
  \centering
  \includegraphics[width=3.0in]{figure-3.png}
  \caption{
    Change of the circuit after tube is lengthened. Red resistors are the original and blue ones are the newly formed. Let $\frac{R_2}{R_1}=(\frac{N_2}{N_1})^{-\lambda}$ where $R$ stands for the resistance and $N$ for the number of contacts, then in (a) and (b) $\lambda=2.7$ as the best case. (c) $\lambda=2.3$. (d) $\lambda=1$ as the worst case.
  }
\end{figure}

\section{numerical simulations}

In the simulation, the network is generated with a large number of random distributed nanotubes using Matlab [Fig. 1(b)]. There are three independent parameters: x y coordinates and the tube angle.
\begin{equation}
  \left\{
  \begin{aligned}
    x_{i}&=L_{ch}\cdot rand_{1i} \\
    y_{i}&=W\cdot rand_{2i} \\
    \alpha_{i}&=2\pi\cdot rand_{3i}
  \end{aligned}
  \right.
\end{equation}
where $(x_i,y_i)$ means the coordinates of the head of each generated tube, and $\alpha_i$ is the tube angle. Hence, the end coordinates of each tube are
\begin{equation}
  \left\{
  \begin{aligned}
    x'_{i}&=x_{i}+l_{cnt}\cdot\cos(\alpha_{i}) \\
    y'_{i}&=y_{i}+l_{cnt}\cdot\sin(\alpha_{i})
  \end{aligned}
  \right.
\end{equation}

As an example, for a $10\times 10$ $\mu$m$^2$ TFT channel with tube density of 10 $\mu$m$^{-2}$, there are 1000 nanotubes in total. These three coordinates of tubes is then stored as two $2\times1000$ matrices and a $1\times1000$ matrix respectively. Next, every tube is marked and compared to all the rest tubes iteratively to see if there is a contact between them. For the $i_{th}$  tube with head coordinates of $(x_i,y_i)$, end coordinates of $(x'_i,y'_i)$ and the angle $\alpha_{i}$, its straight-line equation is expressed as
$$y_i=k_ix_i+b_i$$
where $k_i=\tan(\alpha_{i})$ is the geometrical slope of the tube, and $b_i=y_i-k_ix_i$ is the intercept of y-axes. Giving another tube marked as $j_{th}$ , it can be solved that the cross point of the these two lines is
\begin{equation}
  \left\{
  \begin{aligned}
    &x_{i,j}=-\frac{b_i-b_j}{k_i-k_j} \\
    &y_{i,j}=-\frac{b_ik_j-b_jk_i}{k_i-k_j} \\
    &\min(x_i,x_j)<x_{i,j}<\max(x_i,x_j)
  \end{aligned}
  \right.
\end{equation}

Through this process, all tube-tube contacts in the region are found out [Fig. 1]. Their position and connection information is stored in a $1000\times1000$ adjacent matrix.

For such a low tubes density in the network structure, it can be seen from the picture that there are many stand-alone tubes with just one connection to all the other tubes or none of them, which means that they do not participate in the current percolation and obviously should not be considered during calculation of the total channel resistance. So in the next step a specific algorithm is implemented to trim off all the useless branches of the network using the matrices of $x$, $y$, and $x_{i,j}$ [Fig. 4]. The network is then translated into a netlist file and computed by HSPICE.

\begin{figure}[!h]
  \includegraphics[width=3.5in]{figure-4.jpg}
  \caption{
    (a) The trimmed network by cutting off the useless branches iteratively. This step usually takes 2-4 times to converge, since some of the branches do not expose until its only connected tube is cut off. (b) and (c) The local zoom-in figure that shows the remained tubes and their contact points. Every red point stands for a tube-tube tunneling barrier. (d) The equivalent circuit of (c). There are three types of crossing contacts in real percolation: head-by-head type (the left one), cross-over type (the middle one) and triple-path type (the right one).
  }
\end{figure}

\begin{figure}[!h]
  \includegraphics[width=3.5in]{figure-5.png}
  \caption{
    Computed total channel resistance with $\beta=10,20,50$ respectively. Every $\beta$ is calculated for 5 times with $L_{ch}$ ranging from 2 to 20 $\mu$m. The straight lines are the linear regressions of each data group.
  }
\end{figure}

In Fig. 5, let $\beta=r_t/r_{cnt}$ where $r_{cnt}=6.45$ k$\Omega$ is the lowest measured quantum contact resistance of a tube with two ideal contacts, and usually considered as the intrinsic resistance of SWNTs [17]. $\beta$ is thus calculated to be 24 for Eq. (1) and 18 for Eq. (2). Here, $\beta$ is set to be 10, 20 and 50 respectively, in order to study the effect that $r_{cnt}$ might have on the network. The results shows good agreements with Eq. (4), proving that $R_c$ has a linear relation on $L_{ch}$ due to the linear resistance of $r_t$. Besides, the slopes of the corresponding lines are 0.88, 1.71, 4.29 k$\Omega/\mu$m respectively. Note that
\begin{equation}
  \begin{aligned}
    0.88:1.71:4.29\approx (10+\delta):(20+\delta):(50+\delta)
  \end{aligned}
\end{equation}
where $0<\delta<1$ is a small quantity representing the ignorable intrinsic resistance of tubes. Thus in a CNT-TFT channel, the tube-tube contact plays an dominant role in conduction since the tube's intrinsic resistance is too small to be accounted for.

\begin{figure}[!h]
  \includegraphics[width=3.5in]{figure-6.jpg}
  \caption{
    (a) Computed contact density with varied $D$ and $l_{cnt}$, the data fit Eq. (8) well but is a little smaller at both ends. This can be explained. When $l_{cnt}<1$ $\mu$m, there are quite a few stand-alone tubes that are trimmed off and not counted into the total number, so data locates lower than the theoretical curve. When $l_{cnt}>1.5$ $\mu$m, an increasing proportion of tubes would reach out off the channel region, causing a lower probability density region [Fig. 2(b)], which lead to smaller values, too. (b) Computed $D_c-D$ curve, it shows good agreements with the curve.
  }
\end{figure}
\begin{figure}[!h]
  \includegraphics[width=3.5in]{figure-7.png}
  \caption{
    Computed $R_c-l_{cnt}$ curve with other parameters set as $L_{ch}=10$ $\mu$m, $\beta=20$ and $D=10$ $\mu$m$^{-2}$.
  }
\end{figure}

By calculating the fitting parameters in Fig. 7(a), $\lambda$ in Eq. (9) is found to be about 2.0. Hence, $R_{ch}$ is finally derived as
\begin{equation}
  R_{ch}\propto\frac{L_{ch}}{(Dl_{cnt})^{4}}
\end{equation}

\section{conclusions}
In this paper, a theoretical analysis on percolation properties of SWNT network has been proposed. The relations between the four parameters $R_{ch}$, $L_{ch}$, $D$ and $l_{cnt}$ is well studied. Firstly, the channel resistance of $R_{ch}$ has a linear relation on channel length $L_{ch}$ due to the linear resistance of tube-tube contacts which plays a dominant role in percolation, regardless of the CNT/metal contact. Secondly, the contact density is calculated as $D_c=D^2l_{cnt}^2/\pi$. Thirdly, the channel resistance is proportional to $L_{ch}/(Dl_{cnt})^4$. All the above theoretical results are in good agreements with the Monte-Carlo simulations. On account of directly computing the Kirchhoff's equations without losing any information of the network and the contacts, the numerical results are credible and easy to repeat, which could have further applications on similar network problems.

\begin{thebibliography}{99}

\bibitem{c01} Kocabas, C., et al. (2005). "Guided growth of large-scale, horizontally aligned arrays of single-walled carbon nanotubes and their use in thin-film transistors." Small 1(11): 1110-1116.
\bibitem{c02} Kang, S. J., et al. (2007). "Printed multilayer superstructures of aligned single-walled carbon nanotubes for electronic, applications." Nano Letters 7(11): 3343-3348.
\bibitem{c03} Sun, D. M., et al. (2011). "Flexible high-performance carbon nanotube integrated circuits." Nature Nanotechnology 6(3): 156-161.
\bibitem{c04} Rouhi, N., et al. (2011). "Fundamental Limits on the Mobility of Nanotube-Based Semiconducting Inks." Advanced Materials 23(1): 94.
\bibitem{c05} Pike, G. E. and C. H. Seager (1974). "Percolation and Conductivity - Computer Study .1." Physical Review B 10(4): 1421-1434.
\bibitem{c06} Simmons, J. G. (1963). "Generalized Formula for the Electric Tunnel Effect between Similar Electrodes Separated by a Thin Insulating Film." Journal of Applied Physics 34(6): 1793.
\bibitem{c07} Spinelli, G., et al. (2012). "Numerical study of electrical behavior in carbon nanotube composites." International Journal of Applied Electromagnetics and Mechanics 39(1-4): 21-27.
\bibitem{c08} Rahman, R. and P. Servati (2012). "Effects of inter-tube distance and alignment on tunnelling resistance and strain sensitivity of nanotube/polymer composite films." Nanotechnology 23(5): 055703.
\bibitem{c09} Soto, M., et al. (2015). "Modeling Percolation in Polymer Nanocomposites by Stochastic Microstructuring." Materials 8(10): 6697-6718.
\bibitem{c10} Lamberti, P., et al. (2014). "Fabrication and Charge Transport Modeling of Thin-Film Transistor Based on Carbon Nanotubes Network." Ieee Transactions on Nanotechnology 13(4): 795-804.
\bibitem{c11} Kocabas, C., et al. (2007). "Experimental and theoretical studies of transport through large scale, partially aligned arrays of single-walled carbon nanotubes in thin film type transistors." Nano Letters 7(5): 1195-1202.
\bibitem{c12} Pimparkar, N., et al. (2005). "Performance assessment of sub-percolating nanobundle network transistors by an analytical model." Ieee International Electron Devices Meeting 2005, Technical Digest: 541-544.
\bibitem{c13} Pimparkar, N., et al. (2007). "Current-voltage characteristics of long-channel nanobundle thin-film transistors: A "bottom-up" perspective." Ieee Electron Device Letters 28(2): 157-160.
\bibitem{c14} Kumar, S., et al. (2005). "Percolating conduction in finite nanotube networks." Phys Rev Lett 95(6).
\bibitem{c15} Kocabas, C., et al. (2007). "Experimental and theoretical studies of transport through large scale, partially aligned arrays of single-walled carbon nanotubes in thin film type transistors." Nano Letters 7(5): 1195-1202.
\bibitem{c16} Bao, W. S., et al. (2012). "Tunneling resistance and its effect on the electrical conductivity of carbon nanotube nanocomposites." Journal of Applied Physics 111(9): 093726.
\bibitem{c17} Carbon nanotube and graphene device physics H.-S. Philip Wong, Deji Akinwande Cambridge University Press, 2011

\end{thebibliography}

\end{document}
